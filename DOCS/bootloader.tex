\documentclass[12pt]{article}

\author{Jesse E.J. op den Brouw\thanks{\texttt{J.E.J.opdenBrouw@hhs.nl}}\\[2ex]The Hague University of Applied Sciences}
\title{Bootloader}
\date{\today\\[2ex]\normalsize\url{https://github.com/jesseopdenbrouw/riscv-minimal}\\[2ex]For design \texttt{riscv-pipe3-csr-md-lic.bootloader}}

\usepackage[a4paper,inner=1.0in,outer=1.2in,top=1in,bottom=1.5in,footskip=0.4in,showframe]{geometry}
\usepackage{graphicx}
\usepackage{xcolor}
\usepackage[charter]{mathdesign}
\usepackage[scale=0.92]{nimbusmono}
\usepackage{booktabs}
\usepackage{mathtools}
\usepackage{rotating}
\usepackage{register}
\usepackage[english]{babel}
\usepackage{longtable}

%% Making captions nicer...
\usepackage[font=footnotesize,format=plain,labelfont=bf,textfont=sl]{caption}
\usepackage[labelformat=simple,font=footnotesize,format=plain,labelfont=bf,textfont=sl]{subcaption}
\captionsetup[figure]{format=hang,justification=centering,singlelinecheck=off,skip=2ex}
\captionsetup[table]{format=hang,justification=centering,singlelinecheck=off,skip=2ex}
\captionsetup[subfigure]{format=hang,justification=centering,singlelinecheck=off,skip=2ex}
\captionsetup[subtable]{format=hang,justification=centering,singlelinecheck=off,skip=2ex}
%% Put parens around the subfig name (a) (b) etc. Needs labelformat simple
\renewcommand\thesubfigure{(\alph{subfigure})}
\renewcommand\thesubtable{(\alph{subtable})}

% Parskip et al.
\usepackage{parskip}
\makeatletter
\setlength{\parfillskip}{00\p@ \@plus 1fil}
\makeatother

\usepackage{textcomp}
\usepackage{listings}
\lstset{
    basicstyle = \ttfamily,
    numbers = left,
    numberstyle=\tiny\color{gray},
    breaklines = true,
    showspaces = false,
    prebreak = \raisebox{-0.5ex}[0ex][0ex]{\color{red}\ensuremath{\hookleftarrow}},
    postbreak = \raisebox{-0.5ex}[0ex][0ex]{\color{red}\ensuremath{\hookrightarrow}},
    upquote = true,
}

\usepackage{tikz}
\usetikzlibrary{backgrounds,shapes,arrows,automata,decorations.pathreplacing}

\usepackage{hyperref}
\hypersetup{colorlinks}

\begin{document}
\maketitle

\vfill
\begin{abstract}
\parskip=0.5\baselineskip
\noindent
The RISC-V RV32IM three-staged pipelined processor has a spin-off that incorporates a hard-coded bootloader. The bootloader is started at reset and waits for 5 seconds to be contacted by an upload program. If contacted, an S-record file is uploaded to the ROM (or RAM, but programs can only started from the ROM). After uploading, the application is either started or the bootloader falls to a simple monitor program.

\noindent
This processor is not intended as a replacement for commercial available processors.
It is intended as a study object for Computer Science students.

\noindent
This is work in progress. Things will certainly change in the future.
\end{abstract}
\vfill

\clearpage
\tableofcontents

\clearpage
\section{Bootloader}
Design \texttt{riscv-pipe3-csr-md-lic.bootloader} incorporates a hard-coded bootloader with an upload and a simple monitor program. The bootloader is placed in a separate ROM starting at address 0x10000000 and has a maximum length of 4 KB (may be extended). The bootloader cannot be overwritten by an upload.

\section{S-record file}
The S-record standard is invented by Motorola in the 1980's. It consists of formatted lines, called records. Each line can be seen as a record. A record starts with \lstinline|S| followed by a single digit. \lstinline|S0| is used as header record. This record is ignored by the bootloader (skipped). \lstinline|S1|,  \lstinline|S2| and \lstinline|S3| are data record using a 2-byte, 3-byte and 4-byte start address respectively. \lstinline|S4| is reserved and skipped by the bootloader. \lstinline|S5| and \lstinline|S6| are count records and are ignored. \lstinline|S7|, \lstinline|S8| and \lstinline|S9| are termination records with a start address incorporated, with 4-byte, 3-byte and 2-byte address respectively. This start address is used by the bootloader to start the application. Records have a checksum at the end, this checksum is ignored by the bootloader.

\section{Startup sequence}
After loading the design in the FPGA, or after resetting the FPGA, the bootloader starts. It presents itself with a welcome string printed via the USART at default 9600 bps. Then the bootloader waits for about 5 seconds before starting the application at address 0x00000000. During these 5 seconds, at half second interval, a \lstinline|*| is printed via the USART. At the same time, the 10 red leds on the DE0-CV board are lit and dimmed on half second interval from left (high led) to right (low led). If a character is received within the five seconds, either a S-record file can be uploaded or the bootloader falls to a simple monitor program.

\section{Uploading an S-record file}
A Motorola S-record file can be uploaded with the \lstinline|upload| program found in the \lstinline|CODE| directory. It is tested on Linux, Windows is currently not supported. S-record files for all RISC-V programs are generated as part of the \lstinline|make| process by the RISC-V \lstinline|objcopy| program. The \lstinline|upload| program is invoked with:

\begin{lstlisting}[language=]
upload -d <device> -t <timeout> -s <sleep> -v -j file
\end{lstlisting}

The default device is \lstinline|/dev/ttyUSB0| which is the first plugged-in USB-to-USART converter. Timeout is the time the \lstinline|uplead| program waits for expected data from the bootloader. The time is set in deciseconds (0.1 seconds) intervals. The default value is 5. Sleep is the time the \lstinline|upload| program waits after transmitting a character tor the bootloader in microseconds intervals. The default value is 0. The option \lstinline|-v| turns on verbose mode. The option \lstinline|-j| instructs \lstinline|upload| to send a ``start application'' command to the bootloader after the S-record file is uploaded. File must be a valid S-record file.

To upload an S-record file, reset the FPGA or program the FPGA design in the FPGA. Then, within the 5 seconds interval, start the \lstinline|upload| program with options and file name supplied. If the \lstinline|upload| programs manages the contact the bootloader, the S-record file will be uploaded. Depending on the size, uploading may take as short as a few seconds to minutes for a large file. As a rule of thumb, about 700 file characters per seconds are send. Make sure that \emph{no} terminal program (e.g. Putty) is active. If the \lstinline|upload| program cannot contact the bootloader, it exits with an error message. If during sending the records, a response from the bootloader is not read, the \lstinline|upload| exits with an error message. This is mostly due to an open terminal connection. To start the application after the upload, supply the \lstinline|-j| option to the \lstinline|upload| program, otherwise the monitor is started.

\section{Using the monitor}
If within the 5 seconds grace period a character is received by the bootloader, the bootloader falls to a simple monitor program. The monitor recognized some simple commands. Each command is terminated by an enter key.

\lstinline|r|

Run the program a address 0x00000000.

\lstinline|rw <address|

Read and print word af address. Address must be on a 4-byte interval. Data is presented in big endian.

\lstinline|dw <address>|

Dump 16 words from memory to the terminal. Address must be on a 4-byte interval. After each word, 4 ASCII characters are printed, if printable. If not printable, a dot is printed. Useful for finding strings in memory. Data is presented in big endian.

\lstinline|n|

Dump next 16 words from memory to the terminal, and ASCII characters.

\lstinline|rw <address> <data>|

Write 4-byte data at address. Address must be on a 4-byte interval.  Data must be in big endian.

\lstinline|h|

A simple help menu is presented.

Note: the capabilities of the monitor may be extended.

\section{Upload protocol}
Uploading an S-record file uses a simple handshake protocol. The \lstinline|upload| program sends a single exclamation mark (\lstinline|!|). The bootloader responds with an question mark (\lstinline|?|) and a newline (\lstinline|\n|). Now each S-record line is transmitted character by character, including the end-of-line termination character (\lstinline|\r| and/or \lstinline|\n|). After a line is processed, the bootloader responds with a question mark and a newline. After all S-record lines are transmitted, the \lstinline|upload| program either sends a \lstinline|J| to start the application, or a \lstinline|#| to start the monitor.

\section{Implications on the hardware design}
The design has a separate ROM that incorporates the bootloader. The original ROM, at address 0x00000000 is extended with a write port, together with the instruction read port and the data read port. In fact, the ROM has become a (program) RAM. Because the FPGA ROMs (and RAMs) can only have two ports (out/out or in/out), the original ROM hardware is duplicated (by the synthesizer). This takes up some onboard RAM blocks, but very few ALMs (cells). The speed decrements by a few MHz.

Note that the ROM can only be (over)written with words on a 4-byte boundary.

\end{document}